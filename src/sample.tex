\section{Examples}
\subsection{Basic}

\begin{frame}{The Umlauts \& ß}
	\begin{itemize}
		\item ß
		\item ä
		\item ö
		\item ü
	\end{itemize}
	\note{Took me about 2 hours :(}
\end{frame}
\begin{frame}{Unicode test}
	\centering
	私は肉おたべません。
\end{frame}

\begin{frame}
	\frametitle{Lipsum}
	\centering
	\lipsum[1]
\end{frame}

\begin{frame}
	\frametitle{Blocks}
	\begin{block}{Normal Block}
		I'm a normal bock!
	\end{block}
	\begin{exampleblock}{Example Block}
		I'm an example bock!
	\end{exampleblock}
	\begin{alertblock}{Alert Block}
		I'm a alert bock!
	\end{alertblock}
\end{frame}

\begin{frame}{Multiple Columns}
    \begin{columns}[c]
        \column{.45\textwidth}
        \begin{enumerate}
            \item Statement
            \item Explanation
            \item Example
        \end{enumerate}

        \column{.5\textwidth}
        \begin{itemize}
			\item Item 1
			\item Item 2
			\item Item 3
		\end{itemize}

    \end{columns}
\end{frame}

\subsection{Advanced}

\begin{frame}{TM}
	$$M = (Z,\Sigma,\Gamma,\delta,z_0,\square,F)$$
\end{frame}

\begin{frame}[fragile]{Example Java Sourcecode}
	 \begin{block}{Java at it's best}
		\begin{lstlisting}
public class Candy {
	 /*
	 *@ public class invariant name != null
	 * UNICODE: äöüß
	 */

	 // snip
	 private String name;
	 public boolean hasSameName(Candy that) {
		 return this.name.equals(that.name);
	 }
 }
		\end{lstlisting}
	 \end{block}
\end{frame}

\begin{frame}{Tabelle}
	\centering
	\begin{table}
		\begin{tabular}{c c c}
			\toprule
			\textbf{Treatments} & \textbf{Response 1} & \textbf{Response 2} \\
			\midrule
			Treatment 1         & 0.0003262           & 0.562               \\
			Treatment 2         & 0.0015681           & 0.910               \\
			Treatment 3         & 0.0009271           & 0.296               \\
			\bottomrule
		\end{tabular}
		\caption{Table caption}
	\end{table}
\end{frame}

\subsection{Citation}

\begin{frame}[Citation]
    \frametitle{Citation}
    An example of the \enquote{cite} command to cite within the presentation:\\~

    This statement requires citation \cite{p1}.
\end{frame}

\begin{frame}{References}
    % Beamer does not support BibTeX so references must be inserted manually as below
    \footnotesize{
        \begin{thebibliography}{99}
            \bibitem[Smith, 2012]{p1} John Smith (2012)
            \newblock Title of the publication
            \newblock \emph{Journal Name} 12(3), 45 -- 678.
        \end{thebibliography}
    }
\end{frame}

\section[Test]{Proof that there is no failure in the tables of content}