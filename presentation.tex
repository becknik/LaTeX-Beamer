\documentclass[compress, 16pt, aspectratio=1610,	utf8, noamsthm, xcolor=dvipsnames,xcolor=tables,xcolor=hyperref,hyperref=unicode]{beamer}	% TODO handout
							% Add the handout parameters ito the subscript braces to generate nicely formatted 2 in 1 handouts

%% Definitions:
\def\github{https://github.com/becknik/}
\def\mail{\href{mailto:st177878@stud.uni-stuttgart.de}{E-Mail~}}
\def\semester{WS 22/23}
\def\tutNR{X} % TODO


% Importing packages:
%% Encoding, fonts & language(s):
\usepackage{mmap}
\usepackage[nomath]{lmodern}
\usepackage[EU1,T1]{fontenc}
\usepackage[ngerman]{babel}
\usepackage[style=german]{csquotes}

% Japanese font support, works only with otf-ipafont installed
\usepackage{xeCJK}	% CJK package is automatically loaded in beamer?
\setCJKsansfont{IPAGothic} % for \sffamily
%\setCJKmainfont{IPAMincho} % for \rmfamily

%% Basic packages:
\usepackage{graphicx} \graphicspath{figures/}
\usepackage{float}
\usepackage{listings}
% enumerate is automatically leaded in presentation Beamer
\usepackage{mathtools}
\usepackage{amssymb}
\usepackage{wasysym}
%\usepackage{beamertexpower}	% For further tinkering?
\usepackage{tikz}	% (Maybe) For background generation

%% Beamer specific:
\usepackage{qrcode}
\usepackage{pgfpages}	% For handout generation & notes support

%% Testing:
\usepackage{lipsum}


%% Listings setup:
\definecolor{keywords}{HTML}{000080}
\definecolor{types}{HTML}{000000}
\definecolor{instances}{HTML}{660E7A}
\definecolor{string}{rgb}{0.6,0,0}
\definecolor{comments}{HTML}{606060}

\lstset{
	% Formatting
	basicstyle={\ttfamily\small},
	language=Java,
	% Syntax Highlighting:
	keywordstyle={\color{keywords}},
	otherkeywords={},
	keywordstyle=[2]{\itshape\color{types}},
	keywords=[2]{
		String,Main,Test,Integer,Long,Double,Char,List,Collections,Collections,Arrays,Array,
		Candy
	},
	identifierstyle={\bfseries\color{instances}},
	stringstyle={\color{string}},
	commentstyle={\color{comments}},
	morecomment=[l]{*},
	% Code Lines formatting:
	numbers=left,
	numberstyle={\footnotesize\color{comments}},
	stepnumber=1,
	xleftmargin=20pt,
	% Misc:
	tabsize=4,
	showspaces=false,
	showstringspaces=false
}
\usepackage{./src/beamertheme}
%\setbeameroption{show notes on second screen}	% This is different from handout generation


%% Setting up title page:
\title[Übungsgr. 14 - Tutorium \tutNR]{Tutorium \tutNR}
\subtitle[PSE]{Programmierung und Softwareentwicklung}
\author{Jannik Becker}
\institute[Uni Stuttgart]{Universität Stuttgart}
\date[\semester]{\vskip0.5cm\footnotesize \today}

\begin{document}
	\frame{\titlepage}

	\mode<beamer>{
		\begin{frame}{Gliederung}
			\tableofcontents
		\end{frame}

		\AtBeginSection{
			\begin{frame}{Gliederung}
				\tableofcontents[currentsection]
			\end{frame}
		}

		\AtBeginSubsection{
			\begin{frame}{Gliederung}
				\tableofcontents[currentsubsection]
			\end{frame}
		}
	}

	\section{Examples}
\subsection{Basic}

\begin{frame}{The Umlauts \& ß}
	\begin{itemize}
		\item ß
		\item ä
		\item ö
		\item ü
	\end{itemize}
\end{frame}
\begin{frame}{Unicode test}
	\centering
	私は肉おたべません。
\end{frame}

\begin{frame}
	\frametitle{Lipsum}
	\centering
	\lipsum[1]
\end{frame}

\begin{frame}
	\frametitle{Blocks}
	\begin{block}{Normal Block}
		I'm a normal bock!
	\end{block}
	\begin{exampleblock}{Example Block}
		I'm an example bock!
	\end{exampleblock}
	\begin{alertblock}{Alert Block}
		I'm a alert bock!
	\end{alertblock}
\end{frame}

\begin{frame}{Multiple Columns}
    \begin{columns}[c]
        \column{.45\textwidth}
        \begin{enumerate}
            \item Statement
            \item Explanation
            \item Example
        \end{enumerate}

        \column{.5\textwidth}
        \begin{itemize}
			\item Item 1
			\item Item 2
			\item Item 3
		\end{itemize}

    \end{columns}
\end{frame}

\subsection{Advanced}

\begin{frame}{TM}
	$$M = (Z,\Sigma,\Gamma,\delta,z_0,\square,F)$$
\end{frame}

\begin{frame}{Tabelle}
	\centering
	\begin{table}
		\begin{tabular}{c c c}
			\toprule
			\textbf{Treatments} & \textbf{Response 1} & \textbf{Response 2} \\
			\midrule
			Treatment 1         & 0.0003262           & 0.562               \\
			Treatment 2         & 0.0015681           & 0.910               \\
			Treatment 3         & 0.0009271           & 0.296               \\
			\bottomrule
		\end{tabular}
		\caption{Table caption}
	\end{table}
\end{frame}

\subsection{Citation}

\begin{frame}[Citation]
    \frametitle{Citation}
    An example of the \enquote{cite} command to cite within the presentation:\\~

    This statement requires citation \cite{p1}.
\end{frame}

\begin{frame}{References}
    % Beamer does not support BibTeX so references must be inserted manually as below
    \footnotesize{
        \begin{thebibliography}{99}
            \bibitem[Smith, 2012]{p1} John Smith (2012)
            \newblock Title of the publication
            \newblock \emph{Journal Name} 12(3), 45 -- 678.
        \end{thebibliography}
    }
\end{frame}

	% Paste code for \begin{document} in here	% Using this might be more tidy & decrease redundancy


	%\section[]{}
	%\subsection[]{}


	\mode<beamer>{
		\section*{Abschluss}
		\makethanks
	}
\end{document}


%% Useful beamer things:

%		\begin{itemize}
%			\item<1-> Text visible on slide 1
%			\item<2> Text visible on slide 2 and not on 3
%			\item<3-> Text visible on slide 3
%		\end{itemize}

%% QR-Code generation:
%		\vskip1cm\hspace*{\fill}\qrcode[height=1.5cm,level=Q]{}\hspace*{1cm}

%% Block:
%		\begin{block}<+->{}
%			text
%		\end{block}

%% Alter block:
%		\begin{alertblock}<+->{}
%			text
%		\end{alertblock}

%% Example block:
%		\begin{examples}<+->{}
%			text
%		\end{examples}

%% Columns:
%		\begin{columns}
%			\column{0.5\textwidth}
%			text
%			\column{0.5\textwidth}
%			text
%		\end{columns}
